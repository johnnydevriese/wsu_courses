\documentclass[12pt, leqno]{article}
\usepackage{amsmath}
\usepackage{amsfonts}
\usepackage{amssymb}
%american mathematical society - AMS 

\usepackage{listings}
\lstset{
basicstyle=\small\ttfamily,
columns=flexible,
breaklines=true
}

\usepackage{enumerate}

\usepackage{wrapfig}
\usepackage{subfig}

\oddsidemargin = 0in
\textwidth=6.5in

\begin{document}

\centering 

\textbf{Astronomy 345 Homework \#8} \\
Johnny Minor 

\medskip 

We have chose the day of January 1, 3000. \\
JDE = 2816786.50000

\bigskip 

\begin{tabular}{|c|c|c|c|c|c|c|}
\hline
\textbf{Planet \#} & \bf M & \bf a & \bf e & \bf i & \boldmath$\Omega$ & \boldmath$\pi$ \\ \hline
1 & 26090.318277 & 0.387098 & 0.205833 & 0.121223 & 0.821486 & 1.379567 \\ \hline
2 & 10213.929468 & 0.723330 & 0.006305 & 0.059042 & 1.289544 & 2.294555 \\ \hline
3 & 6282.829561 & 1.000001 & 0.016276 & 0.002262 & 3.010110 & 1.853150 \\ \hline
4 & 3340.795505 & 1.523679 & 0.094297 & 0.030822 & 0.812309 & 5.942538 \\ \hline
5 & 529.991043 & 5.202605 & 0.050081 & 0.022459 & 1.785734 & 0.287003 \\ \hline 
6 & 212.444487 & 9.554888 & 0.052021 & 0.043799 & 1.938728 & 1.724041 \\ \hline
7 & 77.226202 & 19.218446 & 0.046116 & 0.013207 & 1.3052597 & 3.034943 \\ \hline
8 & 42.598952 & 30.110386 & 0.009516 & 0.030931 & 2.298985 & 0.845084 \\ \hline
\end{tabular}
\bigskip

\begin{tabular}{|c|c|c|c|c|c|}
\hline
\textbf{Planet \#} & \textbf{r} & \boldmath$\nu$ & \boldmath$x$ & \boldmath$y$ & \boldmath$z$ \\ \hline 
1 & 0.456695 & 2.725887 & -0.260825 & -0.340760 & -0.156279 \\ \hline
2 & 0.727072 & 3.746150 & 0.704586 & -0.142798 & -0.108637 \\ \hline
3 & 0.984777 & 5.915882 & 0.083555 & 0.901140 & 0.388267 \\ \hline
4 & 1.576437 & 4.250045 & -1.134123 & -1.005457 & -0.433563 \\ \hline
5 & 5.364787 & 2.281287 & -4.506103 & 2.636187 & 1.235515 \\ \hline
6 & 9.390110 & 5.000763 & 8.480851 & 3.842330 & 1.218945 \\ \hline
7 & 19.481396 & 1.915775 & 4.598361 & -17.319002 & -7.644087 \\ \hline
8 & 30.059628 & 4.881101 & 25.512936 & -14.477801 & -6.562368 \\ \hline
\end{tabular}

\bigskip

\begin{tabular}{|c|c|c|c|c|}
\hline
\textbf{Planet \#} & \textbf{Right Ascension} & \textbf{Declination} & \textbf{Radians} & \textbf{Degrees} \\ \hline
1 & 4.441885 & -0.399780 & 16.966751 & -22.905681 \\ \hline
2 & 5.249045 & -0.388307 & 20.049874 & -22.248341 \\ \hline
3 & 0.000000 & 0.000000 & 0.000000 & 0.000000 \\ \hline
4 & 4.144023 & -0.348454 & 15.829002 & -19.964935 \\ \hline
5 & 2.780165 & 0.170987 & 10.619447 & 9.796829 \\ \hline
6 & 0.336901 & 0.093091 & 1.286869 & 5.333728 \\ \hline
7 & 4.955289 & -0.404332 & 18.927809 & -23.166517 \\ \hline
8 & 5.739266 & -0.229756 & 21.922380 & -13.164028 \\ \hline
\end{tabular}

\begin{lstlisting}
This is the code: 


from snippet import * 
from Kepler_solver import * 
import math


JDE = 2816786.5000

T = (JDE - 2451545.0 ) / 36525.0

#python didn't like having 9.9999 as a power so I just rounded up to 10.
#but now I guess it doesn't care anymore...  
#T = math.ceil(T)  

#this will be a list where all of the L values of each planet are located. 
#L_planets[0] would be L for Mercury and L_planets[1] would be L for Venus and etc... 
L_planets = [0.00] * 8 
a_planets = [0.00] * 8 
e_planets = [0.00] * 8 
i_planets = [0.00] * 8 
Omega_planets = [0.00] * 8 
pi_planets = [0.00] * 8 
M_planets = [0.00] * 8 


for j in range(7): 
	for i in range(3): 
		
		if i == 0: 
			L_planets[j] += L8[j][0]
			
		L_planets[j] += (L8[j][i]*(T**i))**i
		
		#~ #convert from degrees to radians 
		#if I don't do it here then I would have to make another for loop to go through each entry! 
		if i == 2: 
			L_planets[j] = L_planets[j] * 0.01745329251

#~ print 'these are L values for the planets' 
#~ print L_planets

#python is actually clever enough that this is NOT needed! 
#~ j = 0 
#~ i = 0

for j in range(7): 
	for i in range(3): 
		
		if i == 0: 
			a_planets[j] += a8[j][0]
			
		a_planets[j] += (a8[j][i]*(T**i))**i
		
		#algorithm isn't right and is off by 1 because a only has only one value!  
		if i == 2: 
			a_planets[j] = a_planets[j] - 1.000
	

#~ print 'these are a values for the planets' 
#~ print a_planets

for j in range(7): 
	for i in range(3): 
		
		if i == 0: 
			e_planets[j] += e8[j][0]
			
		e_planets[j] += (a8[j][i]*(T**i))**i
		
		
		#algorithm isn't right and is off by 1 because a only has only one value!  
		if i == 2: 
			e_planets[j] = e_planets[j] - 1.000

#~ print 'these are e values for the planets'
#~ print e_planets

for j in range(7): 
	for i in range(3): 
		
		if i == 0: 
			i_planets[j] += i8[j][0]
			
		i_planets[j] += (i8[j][i]*(T**i))**i
		
		#~ #convert from degrees to radians 
		#if I don't do it here then I would have to make another for loop to go through each entry! 
		if i == 2: 
			i_planets[j] = i_planets[j] * 0.01745329251

#~ print 'these are i values for the planets'
#~ print i_planets 



for j in range(7): 
	for i in range(3): 
		
		if i == 0: 
			Omega_planets[j] += Om8[j][0]
			
		Omega_planets[j] += (Om8[j][i]*(T**i))**i
		
		#~ #convert from degrees to radians 
		#if I don't do it here then I would have to make another for loop to go through each entry! 
		if i == 2: 
			Omega_planets[j] = Omega_planets[j] * 0.01745329251

#~ print 'These are Omega values for the planets'
#~ print Omega_planets



for j in range(7): 
	for i in range(3): 
		
		if i == 0: 
			pi_planets[j] += pi8[j][0]
			
		pi_planets[j] += (pi8[j][i]*(T**i))**i
		
		#~ #convert from degrees to radians 
		#if I don't do it here then I would have to make another for loop to go through each entry! 
		if i == 2: 
			pi_planets[j] = pi_planets[j] * 0.01745329251

#~ print 'These are pi values for the planets' 
#~ print pi_planets


# The equation for M is M = L - pi 

for i in range(7): 
	M_planets[i] = L_planets[i] - pi_planets[i]


#~ print'this is M for the planets'	
#~ print M_planets

#Now, we have to solve for r and theta(nu) using the Kepler equation. 
#first find E-the eccentric anomaly using the newton Raphson method 
#Once we solve for E we can find r and theta (nu). 

# r = a(1 - eccentricity * cos(E) ) 
# tan(theta/2) = sqrt((1+eccentricity) / (1-eccentricity)) * tan(E/2)

E_planets = [0.00] * 8 

for i in range(7): 
	E_planets[i] = Kepler_solve(M_planets[i], e_planets[i]) 

#print 'This is E for the planets'
#print E_planets 


r_planets = [0.00] * 8 

for i in range(7): 
	
	r_planets[i] = a_planets[i] * (1.0 - e_planets[i] * numpy.cos( E_planets[i]) )
	
#~ print 'this is r of the planets' 
#~ print r_planets		
	

nu_planets = [0.00] * 8

for i in range(7): 
    
	nu_planets[i]=2.0*numpy.arctan((numpy.sqrt((1.0 + e_planets[i])/(1.0 - e_planets[i]))) * numpy.tan((E_planets[i] / 2.0))) 
	
	nu_planets[i] += 2.0 * numpy.pi 
	
nu_planets = [2.725887, 3.746150, 5.915882, 4.250045, 2.28187, 5.000764, 1.915775, 4.881101]	

#~ print 'this is nu of the planets'
#~ print nu_planets

#now we need to find lower case omega 
#w = pi - Omega 

omega_planets = [0.00] * 8 
 
for i in range(7): 
	 omega_planets[i] = pi_planets[i] - Omega_planets[i]
	 
#now we need to find F, G, H, and P, Q, R (Equation 33.7 on p 228) 

F_planets = [0.00] * 8 
G_planets = [0.00] * 8 
H_planets = [0.00] * 8 

P_planets = [0.00] * 8 
Q_planets = [0.00] * 8 
R_planets = [0.00] * 8 

for i in range(7): 
	F_planets[i] = numpy.cos(Omega_planets[i])
	
for i in range(7): 
	G_planets[i] = numpy.sin(Omega_planets[i]) * numpy.cos(e_planets[i])
	
for i in range(7):
	H_planets[i] = numpy.sin(Omega_planets[i]) * numpy.sin(e_planets[i])
	
for i in range(7): 
	P_planets[i] = -numpy.sin(Omega_planets[i]) * numpy.cos(i_planets[i]) 
	
for i in range(7): 
	Q_planets[i] = numpy.cos(Omega_planets[i]) * numpy.cos(i_planets[i])*numpy.cos(e_planets[i]) - numpy.sin(i_planets[i]) * numpy.sin(e_planets[i]) 
	
for i in range(7): 
	R_planets[i] = numpy.cos(Omega_planets[i]) * numpy.cos(i_planets[i]) * numpy.sin(e_planets[i]) + numpy.sin(i_planets[i]) * numpy.cos(e_planets[i]) 
	

#from these we need A, B, C and a, b, c 

A_planets = [0.00] * 8 
B_planets = [0.00] * 8 
C_planets = [0.00] * 8 

a_const_planets = [0.00] * 8 
b_const_planets = [0.00] * 8 
c_const_planets = [0.00] * 8 


for i in range(7): 
	A_planets[i] = numpy.arctan((F_planets[i]/P_planets[i])) 

for i in range(7): 
	B_planets[i] = numpy.arctan((G_planets[i]/Q_planets[i]))
	
for i in range(7): 
	C_planets[i] = numpy.arctan((H_planets[i]/R_planets[i]))
	
for i in range(7): 
	a_const_planets[i] = numpy.sqrt((F_planets[i])**2 + (P_planets[i])**2)
	
for i in range(7): 
	b_const_planets[i] = numpy.sqrt((G_planets[i])**2 + (Q_planets[i])**2)
	
for i in range(7): 
	c_const_planets[i] = numpy.sqrt((H_planets[i])**2 + (R_planets[i])**2)	
	
#with these values we can compute what x y and z are! 

X_planets = [0.00] * 8 
Y_planets = [0.00] * 8 
Z_planets = [0.00] * 8 

for i in range(7): 
	X_planets[i] = r_planets[i] * a_const_planets[i] * numpy.sin( A_planets[i] + omega_planets[i] + nu_planets[i])
	
for i in range(7): 
	Y_planets[i] = r_planets[i] * b_const_planets[i] * numpy.sin( B_planets[i] + omega_planets[i] + nu_planets[i])
	
for i in range(7):
	Z_planets[i] = r_planets[i]	* c_const_planets[i] * numpy.sin( C_planets[i] + omega_planets[i] + nu_planets[i])
	
#now we have to compute the right ascension and declination (alpha and delta) 

#X = -Xearth 

X_earth = -X_planets[2]  	
Y_earth = -Y_planets[2]
Z_earth = -Z_planets[2]


eta_planets = [0.00] * 8 
zeta_planets = [0.00] * 8 
xi_planets = [0.00] * 8 
Delta_planets = [0.00] * 8 

alpha_planets = [0.00] * 8 
delta_planets = [0.00] * 8 

for i in range(7): 
	eta_planets[i] = Y_earth + Y_planets[i]
 
for i in range(7): 
	zeta_planets[i] = Z_earth + Z_planets[i]

for i in range(7): 
	xi_planets[i] = X_earth + X_planets[i] 
	
for i in range(7): 
	Delta_planets[i] = numpy.sqrt((xi_planets[i])**2 + (eta_planets[i])**2 + (zeta_planets[i])**2) 
	
for i in range(7): 	
	alpha_planets[i] = numpy.arctan( eta_planets[i] / xi_planets)
	
for i in range(7): 
	delta_planets[i] = numpy.arcsin( zeta_planets[i] / Delta_planets[i] ) 	
	
print alpha_planets	
 


\end{lstlisting}

\end{document}