\documentclass[11pt, leqno]{article}
\usepackage{amsmath}
\usepackage{amsfonts}
\usepackage{amssymb}
%american mathematical society - AMS 

\addtolength{\topmargin}{-.875in}
	\addtolength{\textheight}{1.75in}

\usepackage{listings}
\lstset{
basicstyle=\small\ttfamily,
columns=flexible,
breaklines=true
}

\usepackage{enumerate}

\usepackage{wrapfig}
\usepackage{subfig}

\oddsidemargin = 0in
\textwidth=6.5in

\begin{document}

{\centering
\textbf{Classical Mechanics (320)} \\  
7.15, 7.20, 7.21, 7.24 \\
Johnny Minor \\
}

\noindent \textbf{7.15}

{\centering
%{\centering
\fbox{ 
\begin{minipage}{4in} 
\hfill\vspace{2in} 
\end{minipage} } \\
}
\bigskip
We begin with the Lagrangian
$$
\mathcal{L} = T - U.
$$
Where $T$ is the kinetic energy and $U$ is the potential energy of the system. 
Since the two masses are connected by a string we can then say that both of the masses will have the same velocity. 
$$
\dot{x_1} = \dot{x_2} = \dot{x}
$$
This also works well since the problem told us to use a generalized coordinate of $x$. So, the total kinetic energy will be a sum of the two masses. 
$$ 
T = \frac{1}{2}m_1\dot{x}^2+\frac{1}{2}m_2\dot{x}^2
$$
The author tells us that we need to have the generalized coordinate of distance $x$ below the top. Therefore, we should set the potential equal to zero at the top of the table. This means that the potential energy of $m_1$ will be 0 and we will only have 
$$
U = -m_2 g x
$$
If we combine the two terms then our Lagrangian will be 
$$
\mathcal{L} = \frac{1}{2}m_1\dot{x}^2+\frac{1}{2}m_2\dot{x}^2 + m_2 g x 
$$
So we can then use the Euler-Lagrange equation to solve for the acceleration of the blocks. 
$$
\frac{\partial \mathcal{L}}{\partial x} - \frac{d}{dt}\left(\frac{\partial \mathcal{L}}{\partial \dot{x}}\right)= 0 
$$
We'll work piecewise working on the first term. We get 
\begin{align*}
\frac{\partial \mathcal{L}}{\partial x} &= \frac{\partial}{\partial x}\left(\frac{1}{2}m_1\dot{x}^2+\frac{1}{2}m_2\dot{x}^2 + m_2 g x \right) \\ 
&= m_2 g \\
\end{align*}
And then for the second term we get 
\begin{align*}
\frac{d}{dt}\left(\frac{\partial \mathcal{L}}{\partial \dot{x}}\right) &= \frac{d}{dt}\left( \frac{\partial}{\partial \dot{x}}\left(\frac{1}{2}m_1\dot{x}^2+\frac{1}{2}m_2\dot{x}^2 + m_2 g x \right) \right) \\
&=  \frac{d}{dt}\left( m_1 \dot{x} + m_2 \dot{x} \right) \\ 
&= (m_1 + m_2)\ddot{x}\\
\end{align*}
So if we then combine the two expression we have 
$$
m_2 g - (m_1 + m_2)\ddot{x} = 0 
$$
Not only is it traditional to solve for $\ddot{x}$, but the problem demands we solve for acceleration. Since $\ddot{x}$ is acceleration I think we can say
$$
\boxed{a = \frac{m_2 g}{m_1 + m_2}}
$$
% ************************* end of 7.15 **********************************************

\bigskip

\noindent \textbf{7.20}
We always start with trying to form the Lagrangian of the system 
$$
\mathcal{L} = T - U
$$
First we will find the position of the bead on the string. In cylindrical polar coordinates it would be defined as 
$$
\vec{r} = (\rho, \phi, z)
$$
The relationships were given in the problem. Solving for $\phi$. We can plug them in to get 
$$
\vec{r} = (R, \frac{z}{\lambda}, z) 
$$
Our aim is to find the kinetic energy expression in generalized coordinates. To do this we need to find an expression for the velocity. Well it will just be $d/dt$ of each term in $r$. However, we know $R$ and $\lambda$ are constants. So our velocity will be 
$$
\vec{v} = (0, \frac{\dot{z}}{\lambda}, \dot{z})
$$
However, we need $v^2$ for our kinetic energy term. That will be 
\begin{align*}
v^2 &= R^2 \dot{\phi}^2 + \dot{z}^2 \\
&= (1 + \frac{R^2}{\lambda^2 }) \dot{z}^2
\end{align*}
We can then plug this into our kinetic energy and find 
\begin{align*} 
T &= \frac{1}{2} m v^2 \\ 
&= \frac{1}{2} m (1 + \frac{R^2}{\lambda^2 }) \dot{z}^2
\end{align*}
Gravity will point down along the $z$ coordinate so the potential energy will simply be 
$$
U = mgz
$$
If we combine $T$ and $U$ we form the Lagrangian. 
$$
\mathcal{L} =  \frac{1}{2} m (1 + \frac{R^2}{\lambda^2 }) \dot{z}^2 - mgz 
$$
We now apply the Euler-Lagrange equation.
$$
\frac{\partial \mathcal{L}}{\partial z} - \frac{d}{dt}\left(\frac{\partial \mathcal{L}}{\partial \dot{z}}\right)= 0 
$$
If we start with the first term.
\begin{align*}
\frac{\partial\mathcal{L} }{\partial z} &= \frac{\partial}{\partial z}\left( \frac{1}{2} m (1 + \frac{R^2}{\lambda^2 }) \dot{z}^2 - mgz  \right) \\ 
&= -mg \\ 
\end{align*}
If we evaluate the second term we have 
\begin{align*}
\frac{d}{dt}\left(\frac{\partial \mathcal{L}}{\partial \dot{z}}\right) &= \frac{d}{dt}\left(\frac{\partial}{\partial \dot{z}} \left(  \frac{1}{2} m (1 + \frac{R^2}{\lambda^2 }) \dot{z}^2 - mgz \right) \right)  \\
&=  \frac{d}{dt}\left( m (1 + \frac{R^2}{\lambda^2 }) \dot{z} \right) \\ 
&= m (1 + \frac{R^2}{\lambda^2 }) \ddot{z} \\ 
\end{align*}
We then recombine the two terms of our Euler-Lagrange. 
$$ 
-mg - m (1 + \frac{R^2}{\lambda^2 }) \ddot{z} = 0
$$ 
We can then solve for $\ddot{z}$
$$
\boxed{\ddot{z} = \frac{-g}{1 + \frac{R^2}{\lambda^2}} }
$$
We examine when $R \rightarrow 0$ and notice that $\ddot{z}$ will be $-g$. This would make sense because we would approach being completely on the $z$ axis and gravity is completely on along that coordinate.  

%************************* END OF 7.20 *************************************************** 
\bigskip
\noindent \textbf{7.21}
We begin with the Lagrangian 
$$ 
L = T - U 
$$ 
We note that the frictionless rod is in a horizontal plane. So it seems logical to choose the potential energy to zero at that point. Therefore we won't have to worry about a potential energy term in our Lagrangian. Since we have two coordinates of $r$ and $\phi$. So the derivative of those terms will be needed. We'll get 
$$
v = \sqrt{\dot{r}^2 + (r\omega)^2 }
$$ 
So we can then say that our kinetic energy term $T$ will be 
$$
T = \frac{1}{2} (m \dot{r}^2 + (r\omega)^2) 
$$ 
So, our Lagrangian will just be 
$$ 
\mathcal{L} = \frac{1}{2} (m \dot{r}^2 + (r\omega)^2) 
$$
We can then use the Euler-Lagrange equation. 
$$
\frac{\partial \mathcal{L}}{\partial r} - \frac{d}{dt}\left(\frac{\partial \mathcal{L}}{\partial \dot{r}}\right)= 0 
$$
We begin with the first term to be 
\begin{align*}
\frac{\partial \mathcal{L}}{\partial r} &= \frac{\partial}{\partial r}\left( \frac{1}{2} (m \dot{r}^2 + (r\omega)^2) \right) \\ 
&= mr\omega^2 \\ 
\end{align*}
The second term will be 
\begin{align*}
\frac{d}{dt}\left(\frac{\partial \mathcal{L}}{\partial \dot{r}}\right) &= \frac{d}{dt}\left(\frac{\partial }{\partial \dot{r}}\left( \frac{1}{2} (m \dot{r}^2 + (r\omega)^2) \right) \right) \\
&= \frac{d}{dt} \left( m\dot{r} \right) \\
&= m \ddot{r}
\end{align*} 
We then combine the two terms which yields 
$$ 
mr\omega^2 - m \ddot{r} = 0
$$ 
We then solve for $\ddot{r}$ to get 
\begin{equation}
\boxed{\ddot{r} = r \omega^2  }
\end{equation}
Now, if we analyze when the bead is initially at rest at the origin then that means $r = 0$ and $\dot{r} = 0$. This would lead to the bead never leaving the origin because $r = 0$, $\dot{r} = 0$ and $\ddot{r} = 0$. So then equation (1) will just be zero too. 

However, if we have $r_0 > 0$ then we know the solution to this differential equation is the familiar 
$$
r(t) = A e^{\omega t} + B e^{-\omega t}
$$
We can then use the initial conditions to find the constants $A$ and $B$. 
\begin{align*}
r_0 &= A  e^{\omega 0} + B e^{-\omega 0} \\
&= A + B 
\end{align*}
We need another initial condition to be able to solve for $A$ and $B$ so we take the derivative since we know that $\dot{r}(0) = 0$. 
$$
\dot{r}(t) = A \omega A e^{\omega t} + B \omega e^{-\omega t}
$$
If we use the initial condition to solve for the constants we have 
\begin{align*}
0 &= A \omega - B \omega  \\
A &= B 
\end{align*}
We can then use both results of the initial conditions to find 
$$
A = B = \frac{r_0}{2}
$$
If we plug these constants back into our general solution we have our equation of motion to be. 
$$ 
\boxed{r(t) = \frac{r_0}{2}e^{\omega t} + \frac{r_0}{2} e^{-\omega t} }
$$
By inspection we can see that this equation will eventually grow exponentially as the $e^{-\omega t}$ will go to zero and the $e^{\omega t}$ will dominate, growing exponentially. 

This results corresponds to a centrifugal force because if you were in the inertial reference frame then $r= 0$ and $\dot{r} = 0$ so wouldn't seem to be moving. Which is just like the first case that we looked at. However, if outside the inertial reference frame then you find a radially outward force like in the second case. 





%******************************** end of problem 7.21 ************************ 
\bigskip 
\noindent \textbf{7.24}
From equation (7.54) we know that the Lagrangian of the Atwood machine is
$$
\mathcal{L} = T - U = \frac{1}{2}(m_1 + m_2)\dot{x}^2 + (m_1 + m_2)gx 
$$ 
We can then apply the Euler-Lagrange equation to find generalized force and generalized momentum. 
$$
\frac{\partial \mathcal{L}}{\partial x} - \frac{d}{dt}\left(\frac{\partial \mathcal{L}}{\partial \dot{x}}\right)= 0 
$$ 
If we start with the first term we have 
\begin{align*}
\frac{\partial \mathcal{L}}{\partial x} &= \frac{\partial}{\partial x}\left( \frac{1}{2}(m_1 + m_2)\dot{x_2}^2 + (m_1 + m_2)gx  \right) \\ 
&= (m_1 - m_2)g \\
\end{align*}
And then we can carry out the derivatives of the second term 
\begin{align*}
\frac{d}{dt}\left(\frac{\partial \mathcal{L}}{\partial \dot{x}}\right) &=  \frac{d}{dt}\left(\frac{\partial }{\partial \dot{x}} \left( \frac{1}{2}(m_1 + m_2)\dot{x^2} + (m_1 + m_2)gx \right) \right) \\
&= \frac{d}{dt}((m_1 + m_2)\dot{x}) \\ 
&= (m_1 + m_2)\ddot{x} \\
\end{align*}

So we know that 
\begin{align*}
\frac{\partial \mathcal{L}}{\partial x} &= \frac{d}{dt}\left( \frac{\partial \mathcal{L}}{\partial \dot{x}} \right) \\ 
\text{generalized force} &= \text{rate of change of generalized momentum} \\
(m_1 - m_2)g  &= (m_1 + m_2)\ddot{x} \\ 
\end{align*}
So, we have found the expected result. We can then solve for $\ddot{x}$ to get 
$$ 
\boxed{\ddot{x} = \frac{(m_1 - m_2)g }{m_1 + m_2}}
$$

\end{document} 