\documentclass[10pt, reqno]{article}
\usepackage{amsmath}
\usepackage{amsfonts}
\usepackage{amssymb}

\newcommand\blfootnote[1]{%
  \begingroup
  \renewcommand\thefootnote{}\footnote{#1}%
  \addtocounter{footnote}{-1}%
  \endgroup
}

\numberwithin{equation}{section}

\begin{document}

%\numberwithin{section}{equation} 


\section*{Principle Vectors}
%\newcounter{section}
\setcounter{section}{39}

\textbf{\arabic{section}.} For a matrix $A$ with linear elementary divisors there exist $n$ eigenvectors
spanning the whole $n$-space. If A has a non-linear divisor, however, this is not
true since there are then fewer than n independent eigenvectors. It is convenient
nevertheless to have a set of vectors which span the whole $n$-space and to choose
these in such a way that they reduce to the $n$ eigenvectors when $A$ has linear
divisors. Now we have seen that in the latter case the eigenvectors may be taken
as the columns of a matrix $X$, such that



%TODO: Get numbering correct! 
\begin{equation}
X^{-1}AX = \textrm{diag}(\lambda_i).
\end{equation}

\noindent A natural extension when the matrix has non-linear divisors is to take, as base
vectors the $n$ columns of a matrix $X$ which reduces $A$ to the Jordan canonical form.

These vectors satisfy important relations. It will suffice to demonstrate them
for a simple example of order 8. Suppose the matrix $A$ is such that

\begin{equation}
AX = A
\begin{bmatrix}
C_3(\lambda_1) & & & \\
& C_2(\lambda_1) & & \\
& & C_2(\lambda_2) & \\
& & & C_1(\lambda_3) \\
\end{bmatrix},
\end{equation}

\noindent then if $x_1, x_2, \dots,x_8$ are the columns of $X$, we have equating columns

%TODO: get some brackets going on the end.

\begin{equation}
	\left.
	\begin{aligned}
	Ax_1 &= \lambda_1 x_1 + x_2, & Ax_4 &= \lambda_1 x_4 + x_5 , & Ax_6 &= \lambda_2 x_6 + x_7, & Ax_8 &= \lambda_3 x_8 \\
	Ax_2 &= \lambda_1 x_2 + x_3, & Ax_5 &= \lambda_1 x_5, & Ax_7 &= \lambda_2 x_7 & & \\
	Ax_3 &= \lambda x_1 & & & \\ 
	\end{aligned}
	\right\} ,	
\end{equation}


\noindent from which we deduce 
\smallskip

\begin{equation}
	\left.
	\begin{aligned}
	 {&(A -\lambda_1I)}^3 x_1 = 0, & {&(A-\lambda_1I)}^2 x_4 = 0, & { &(A-\lambda_2I)}^2 x_6 = 0, & &(A-\lambda_3 I) x_8 = 0 \\
	{ &(A-\lambda_1I)}^2 x_2 = 0, & &(A - \lambda_1I)x_5 = 0, & &(A-\lambda_2I)x_7 = 0 &  \\
	 &(A-\lambda_1I)x_3 = 0 & & & \\ 
	\end{aligned}
	\right\} .
\end{equation}

Each of these vectors therefore satisfies a relation of the form

\begin{equation}
{(A-\lambda_i I)}^j x_k = 0. 
\end{equation}

%could change the (39.5) to an actual counter? 
A vector which satisfies equation (39.5) but does not satisfy a relation of lower degree in $(A-\lambda_i I)$ is called a \textit{principal vector of grade j} corresponding to $\lambda_i$.

%this relies on the special command blfootnote defined in the preamble. 
\blfootnote{From J. H. Wilkinson, \textbf{The Algebraic Eigenvalue Problem}, Oxford University Press, Oxford,
1965.}

\end{document}